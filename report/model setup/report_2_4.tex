\documentclass[11pt]{article}

\usepackage{natbib}
\bibliographystyle{unsrtnat}

% change document font family to Palatino, and code font to Courier
\usepackage{mathpazo} % add possibly `sc` and `osf` options
%\usepackage{eulervm}
%\usepackage{courier}
%allow formula formatting

% flow chart
\usepackage{tikz}
\usetikzlibrary{shapes.geometric, arrows}
\usetikzlibrary{bayesnet}

%identation in nested enumerates
\usepackage[shortlabels]{enumitem}
\setlist[enumerate,1]{leftmargin=1cm} % level 1 list
\setlist[enumerate,2]{leftmargin=2cm} % level 2 list

%flush align equations to left, this also loads amsmath 
%\usepackage[fleqn]{mathtools}
\usepackage{mathtools}
\usepackage{amsthm}
\DeclareMathAlphabet\mathbfcal{OMS}{cmsy}{b}{n}
\usepackage{comment}

% allow algorithm 
\usepackage[linesnumbered,ruled,vlined]{algorithm2e}
\newcommand\mycommfont[1]{\ttfamily\textcolor{blue}{#1}}
\SetCommentSty{mycommfont}

%allow equation numbering in {align*} envir
\newcommand\numberthis{\addtocounter{equation}{1}\tag{\theequation}}

%declare math symbolz
%# inner product
\DeclarePairedDelimiter{\inner}{\langle}{\rangle}

%declare argmin
\newcommand{\argmin}{\operatornamewithlimits{argmin}}

%declare checkmark
\usepackage{amssymb}
\usepackage{pifont}% http://ctan.org/pkg/pifont
\newcommand{\cmark}{\ding{51}}%
\newcommand{\xmark}{\ding{55}}%

%title positon
\usepackage{titling} %fix title
\setlength{\droptitle}{-6em}   % Move up the title 

%change section title font size
\usepackage{titlesec} 
\titleformat{\section}
  {\normalfont\fontsize{12}{15}}{\thesection}{1em}{}
\titleformat{\subsection}
  {\normalfont\fontsize{12}{13}}{\thesubsection}{1em}{}
\titleformat{\subsubsection}
  {\normalfont\fontsize{12}{13}}{\thesubsubsection}{1em}{}

%overwrite bfseries to allow formula in section title  
\def\bfseries{\fontseries \bfdefault \selectfont \boldmath}

% change page margin
\usepackage[margin=0.8 in]{geometry} 

%disable indentation
\setlength\parindent{0pt}

%allow inserting multiple graphs
\usepackage{graphicx}
\usepackage[skip=1pt]{subcaption}
\usepackage[justification=centering,font=small]{caption}
\newcommand{\indep}{\rotatebox[origin=c]{90}{$\models$}}%indep sign

%allow code chunks
\usepackage{listings}
%\lstset{basicstyle=\footnotesize\ttfamily,breaklines=true}
\lstset{basicstyle=\footnotesize\ttfamily,breaklines=true}
\lstset{frame=lrbt,xleftmargin=\fboxsep, xrightmargin=-\fboxsep}
\lstset{language=R, commentstyle=\bfseries, 
keywordstyle=\ttfamily} %R-related formatting
\lstset{escapeinside={<@}{@>}}

%allow merged cell in tables
\usepackage{multirow}

%allow http links
\usepackage{hyperref}

%allow different font colors
\usepackage{xcolor}

%Thm and Def environment
\theoremstyle{definition}
\newtheorem{theorem}{Theorem}[section]
\newtheorem{lemma}[theorem]{Lemma}
\newtheorem{proposition}[theorem]{Proposition}
\newtheorem{corollary}[theorem]{Corollary}
\newtheorem{definition}[theorem]{Definition}

\newenvironment{definition2}[1][Definition]{\begin{trivlist} %def without index
\item[\hskip \labelsep {\bfseries #1}]}{\end{trivlist}}

\newenvironment{example}[1][Example]{\begin{trivlist} %def without index
\item[\hskip \labelsep {\bfseries #1}]}{\end{trivlist}}

%allow strikeout effect
\usepackage[normalem]{ulem}

%macros from Bob Gray
\usepackage{"./macro/GrandMacros"}
\usepackage{"./macro/Macro_BIO235"}

\begin{document}
%%%%%%%%%%%%%%%%%%%%%%%%%%%%%%%%%%%%%%%%%%%%%
%%%%%%%%%%%% TItle page with contents %%%%%%%%%%%%%%%
%%%%%%%%%%%%%%%%%%%%%%%%%%%%%%%%%%%%%%%%%%%%%

\title{Progress Report \\ Model Setup \\ \today \vspace{-1ex}}

\pretitle{\begin{flushright}\normalsize}
\posttitle{\par\end{flushright}}
\author{}
\date{}
\vspace{-10em}
\maketitle
\vspace{-5em}

%\tableofcontents 
%\vspace{10em}
%%%%%%%%%%%%%%%%%%%%%%%%%%%%%%%%%%%%%%%%%%%%%
%%%%%%%%%%%% Formal Sections %%%%%%%%%%%%%%%%%%%%%%%%
%%%%%%%%%%%%%%%%%%%%%%%%%%%%%%%%%%%%%%%%%%%%%

\section{\textbf{Simulated Data}}
First generate data from a spatial gaussian process with 100 locations, we sample spatial locations $(x, y)$ iid from standard normal, and assume the pollutant $z$ follow below Gaussian Process:
\begin{align*}
z(x, y) &\stackrel{iid}{\sim} N(f(x, y), \sigma^2 = 0.1) \\
f(x, y) &= 0.2 x + 0.5 y + \sqrt{x^2 + y^2} + sin(x) + cos(x)
\end{align*}

\begin{figure}[!ht]
\begin{subfigure}{.5\textwidth}
  \centering
  \includegraphics[width=.8\linewidth]{"./plot/loc_site"}
  \caption{Sampled saptial location for monitoring sites (standardized)}
  \label{fig:sfig1}
\end{subfigure}%
\begin{subfigure}{.5\textwidth}
  \centering
  \includegraphics[width=\linewidth]{"./plot/y_surface"}
  \caption{Average pollutant surface over space (standardized)}
  \label{fig:sfig2}
\end{subfigure}
%\caption{plots of....}
\label{fig:fig}
\end{figure}

We then generate prediction for $z(x, y)$ from 5 base GP models, with covariance structure:
\begin{enumerate}
\item Linear, 
\item Polynomial, degree 3
\item Gaussian RBF, with ARD
\item Mat\'{e}rn $\frac{5}{2}$, with ARD
\item MLP, with ARD.\\ 
Equivalent to a 2-layer network with Gaussian CDF activation function and infinite hidden units:
$$k(x,y) = \sigma^{2}\frac{2}{\pi }  \text{asin} \left ( \frac{ \sigma_w^2 x^\top y+\sigma_b^2}{\sqrt{\sigma_w^2x^\top x + \sigma_b^2 + 1}\sqrt{\sigma_w^2 y^\top y + \sigma_b^2 +1}} \right )$$
\end{enumerate} 
The out-of-sample MSE for 5 models are $4.846$, $2.094$, $2.011$, $1.989$, $1.981$ respectively.



\clearpage
\bibliography{../report}

\end{document}

%%%%%%%%%%%%%%%%%%%%%%%%%%%%%
%%%%%%%%%%%%%%%%%%%%%%%%%%%%%
%%%%%%%%%%%%%%%%%%%%%%%%%%%%%
%%%%%%%%%%%%%%%%%%%%%%%%%%%%%
%%%%%%%%%%%%%%%%%%%%%%%%%%%%%
%%%%%%%%%%%%%%%%%%%%%%%%%%%%%
%%%%%%%%%%%%%%%%%%%%%%%%%%%%%
%%%%%%%%%%%%%%%%%%%%%%%%%%%%%
